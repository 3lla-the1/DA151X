\documentclass[a4paper,10pt,twocolumn]{article}


\usepackage[utf8]{inputenc} % Support for UTF-8 encoding
\usepackage[T1]{fontenc}    % Better font rendering
\usepackage{graphicx}       % For including images
\usepackage{tabularx}
\usepackage[colorlinks=true, linkcolor=black, citecolor=black, filecolor=black, urlcolor=blue]{hyperref}
\usepackage{geometry}       % To adjust margins
\geometry{margin=2.5cm}
\usepackage{titlesec}       % For custom section headings
\usepackage{fancyhdr}       % For header and footer customization
\usepackage{natbib}         % For in-text citations (Harvard style)
\usepackage{lmodern}        % Improved font quality
\usepackage{setspace}       % For adjusting line spacing
\usepackage{ragged2e}       % Fix justification issues
\usepackage[protrusion=true,expansion=true]{microtype} % Better text spacing
\usepackage{longtable}
\usepackage{amsmath}        % Provides \numberwithin
\usepackage{makecell}
\usepackage{enumitem}
\numberwithin{figure}{section}
\numberwithin{table}{section}
\renewcommand{\thefigure}{\thesection-\arabic{figure}}

% \usepackage{parskip}   % This removes indentation and adds spacing between paragraphs
\setlength{\parindent}{1em}  % First-line indentation for paragraphs
\setlength{\parskip}{0pt}     % No extra space between paragraphs

% ---- Fix fancyhdr warning: set proper headheight ----
\setlength{\headheight}{14.5pt}

% ---- Header settings ----
\pagestyle{fancy}
\fancyhf{}
\rhead{KTH}
\cfoot{\thepage}

% ---- SUB SBUB no stretch settings ----
\usepackage{xspace} % Ensures spacing works correctly
\titleformat{\subsubsection}
  {\normalfont\normalsize\bfseries} % Normal font size, bold
  {\thesubsubsection} % Keep numbering
  {1em} % Indentation before text
  {\raggedright} % Prevent text stretching

\titlespacing{\subsubsection}{0pt}{0pt}{0pt} % Remove extra spacing


% --------------------------------------------------------------------------------------------------------------------------------------------


\begin{document}


% ---- Suppress page numbering on title page to avoid duplicate "page.1" link ----
\clearpage
\pagenumbering{gobble}  % Removes the page number but avoids identifier conflicts

\setlength{\headheight}{14.5pt}
% ---- Title Page ----
\begin{titlepage}
    \centering
    \includegraphics[width=0.2\textwidth]{kthLogga.png}\\[1cm]
    {\large BACHELOR'S THESIS IN COMPUTER SCIENCE AND INDUSTRIAL ECONOMICS}\\[0.5cm]
    {\large UNDERGRADUATE LEVEL 15 CREDITS}\\[3cm]
    {\Huge \textbf{A Comparative Evaluation of Open-Source Digital Asset Management Systems}}\\[0.5cm]
    {\Large Exploring Organizational and Marketing Criteria for Process and Marketing Innovation in SMEs}\\[1cm]
    \vfill
    {\Large \textbf{ELLA KARLSSON}}\\[1cm]
    \vfill
    {\large School of Industrial Engineering and Management}\\
    {\large Royal Institute of Technology (KTH)}\\
\end{titlepage}

% --------------------------------------------------------------------------------------------------------------------------------------------

% ===== Section 1: abstract och sammanfattning... =====
\newpage
\pagenumbering{arabic}
\onecolumn
\phantomsection 
\section*{Abstract}
\addcontentsline{toc}{section}{Abstract}

\citep{author2025}

•	What is the topic area? (optional) Introduces the subject area for the project.
•	Short problem statement
•	Why was this problem worth a Master’s thesis project? (i.e., why is the problem both significant and of a suitable degree of difficulty for a Master’s thesis project? Why has no one else solved it yet?)
•	How did you solve the problem? What was your method/insight?
•	Results/Conclusions/Consequences/Impact: What are your key results/conclusions? What will others do based upon your results? What can be done now that you have finished - that could not be done before your thesis project was completed?

\vspace{0.3cm} 
\textbf{Keywords:} 
    
Digital Asset Management (DAM), Version Control, Metadata Management, Access Control, SMEs, Workflow Optimization

\phantomsection 
\section*{Sammanfattning}
\addcontentsline{toc}{section}{Sammanfattning}

\vspace{0.3cm} 
\textbf{Nyckelord:} 
\newpage

\phantomsection 
\section*{Acknowledgments}
\addcontentsline{toc}{section}{Acknowledgments}
I would like to thank xxxx for having yyyy.

% ---- Table of Contents, list of tables, figures ----
\newpage
\phantomsection 
\onecolumn
\tableofcontents
\newpage

\newpage
\phantomsection 
\addcontentsline{toc}{section}{List of Figures}
\listoffigures
\newpage
\phantomsection 
\addcontentsline{toc}{section}{List of Tables}
\listoftables

% ---- List of Acronyms and Abbreviations ----
\newpage
\phantomsection
\addcontentsline{toc}{section}{List of Acronyms and Abbreviations}
\section*{List of Acronyms and Abbreviations}
\vspace{0.2cm} % Add some space to align table with the heading

\renewcommand{\arraystretch}{1.2} % Adjust row height for readability
\begin{flushleft} % Align the table properly with the heading
\begin{longtable}{p{5cm} p{12cm}} % Adjust column width for proper alignment
    
    AI   & Artificial Intelligence \\
    DAM  & Digital Asset Management \\
    DSR  & Design Science Research \\
    DT   & Digital Transformation \\
    ERP  & Enterprise Resource Planning \\
    IT   & Information Technology \\
    ML   & Machine Learning \\
    MCS  & Management Control Systems \\
    MDM  & Metadata Management \\
    RBAC & Role-based access control \\
    RBV  & Resource-Based View \\
    SME  & Small and Medium-sized Enterprises \\
    UX   & User Experience \\
    VRIN & Valuable, Rare, Inimitable, Non-substitutable \\
    YOLO & You Only Look Once \\
    
\end{longtable}
\end{flushleft}


% --------------------------------------------------------------------------------------------------------------------------------------------

% ---- Switch Back to Two-Column Layout for Main Content ----
\twocolumn
\flushbottom % ÄNDR


% ===== Apply Spacing Fixes =====
% \RaggedRight   % Disable full justification to avoid stretched text
\sloppy        % Allow better text breaking to prevent large spaces
\hyphenpenalty=500
\tolerance=1000


% ===== Section 2: Introduction =====

\section{Introduction}

To be added 
% This chapter describes the specific problem that this thesis addresses, the context of the problem,
% the goals of this thesis project, and outlines the structure of the thesis.
% Give a general introduction to the area. (Remember to use appropriate references in this and all
% other sections.)

\subsection{Background}
Digital Asset Management (DAM) emerged in the late 1990s as organizations began grappling 
with the rapid increase in digital content \citep{krogh2009}. Early DAM systems were primarily on-premises 
solutions designed to store and manage assets such as images, videos, and documents. 
In the early 2000s, these systems transitioned to cloud-based platforms, offering 
improved scalability and accessibility \citep{mccain2021}.

More recently, the integration of Artificial Intelligence (AI) and machine learning (ML) has transformed 
DAM by automating key processes like image tagging, sorting, and categorization. Advanced computer 
vision techniques now enable systems to analyze and tag images automatically, 
reducing manual effort and increasing accuracy \citep{MINGfANG}.

% \begin{figure}[htbp]
%     \centering
%     \includegraphics[width=0.95\linewidth]{DAMC.png}  % Replace with actual image file
%     \caption{Illustrating components of a modern DAM system}
%     \label{fig:DAMC}  
% \end{figure}

% Present the background for the area. Set the context for your project – so that your reader can
% understand both your project and this thesis. (Give detailed background information in Chapter 2 -
% together with related work.)
% Sometimes it is useful to insert a system diagram here so that the reader knows what are the
% different elements and their relationship to each other. This also introduces the names/terms/…
% that you are going to use throughout your thesis (be consistent). This figure will also help you later
% delimit what you are going to do and what others have done or will do.
% As one can find in RFC 1235 [1] multicast is useful for xxxx

\subsection{Problem}
As bespoke manufacturers scale, managing digital assets—spanning product imagery, design renderings,
and technical specifications—becomes essential for brand consistency and operational efficiency.
However, most DAM solutions, especially open-source systems, lack the necessary automation, 
posing adoption and maintenance challenges for small and medium-sized enterprises (SMEs) with limited IT infrastructure. 
Wu et al. studied automated metadata annotation for cultural heritage and found that AI-generated 
captions often oversimplify context, such as describing a medieval knight merely as a “man on a horse” \citep{MINGfANG} 
This reflects similar challenges in design-driven manufacturing, where internal product terminology and industry-specific 
references require more precise and context-aware interpretation.

A core function of DAM is image tagging, sorting, and categorization, directly influencing asset 
retrievability and structural organization. Although AI has been integrated into some DAM solutions,
 these implementations typically rely on large pre-trained models that offer broad object classification 
 rather than domain-specific tagging and vocabulary. Recent advancements in computer vision, 
 particularly through algorithms such as YOLO (You Only Look Once), 
offer an opportunity to overcome these limitations. However, deploying a YOLO-powered system in this domain
 requires adapting the model to the specific features and vocabulary of the manufacturing sector. Rather than
  training a model from scratch—a process that demands extensive annotated data and computational resources—a 
  more feasible approach is to fine-tune a pre-trained model using company-specific data. 


\subsection{Purpose}
% State the purpose of your thesis and the purpose of your degree project.
% Describe who benefits and how they benefit if you achieve your goals. Include anticipated
% ethical, sustainability, social issues, etc. related to your project. (Return to these in your reflections
% in Section 6.4.) /
The primary aim of this thesis is to assess the feasibility and impact of a YOLO-powered DAM system 
that has been fine-tuned on company-specific data to address the unique needs of premium manufacturing SMEs. 
The research will benchmark the performance of this fine-tuned system against a conventional open-source DAM 
platform (ResourceSpace), focusing on improvements in asset categorization accuracy
and retrieval efficiency. 

\vspace{0.3cm}
\subsubsection{Technical Research questions (DATA)}
\vspace{0.3cm}

\begin{enumerate}[label=(\alph*)]
    \item To what extent does fine-tuning YOLOv11 on company-specific data improve metadata accuracy in DAM for manufacturing assets?
    Can it effectively capture the subtle distinctions of assets? 
    
    \item What are the trade-offs between the YOLOv11 model and ResourceSpace DAM tagging methods?
\end{enumerate}

\subsubsection{Business Research questions (INDEK)}
\vspace{0.3cm}
Technological advancements alone do not guarantee successful integration. To complement this,
the business perspective assesses the organizational and strategic impact after selecting the preferred DAM system. Specifically:

\begin{enumerate}[label=(\alph*), resume]
    \item How does employee adaptation, the necessity of training, and any role adjustments impact a bespoke manufacturing company? 
    \item (add / incorporate something about process innovation? )
    \item In what way does improved tagging strangthen brand consistency, customer engagement, and scalability?
\end{enumerate}   

\subsubsection{Societal Impact}
\vspace{0.2cm}
Digital transformation has a significant impact on SMEs.
These companies account for approximately 60\% of total turnover and value-added 
contributions in Sweden’s private sector, employing around 65\% of the 
workforce \citep{tillvaxtverket2021}.
The adoption of DAM systems is an integral part of this transformation, 
improving operational efficiency and reducing manual work,
which contributes to broader economic growth. A cost-benefit analysis of 319 SMEs 
found that digital transformation enhances organizational resilience, reduces 
operational costs, and improves long-term scalability \citep{teng2022}.

\vspace{0.3cm}
The stakeholders of this project?

This study is structured around a systematic process 
encompassing data collection, annotation, model fine-tuning, and testing. 
These phases represent essential steps that an SME would need to undertake 
if they were to implement a similar AI-based solution. 
By addressing both the positive impacts and the possible challenges, the aim is to
to show if the benefits of adopting this solution
justify the necessary investments and efforts.
The project’s outcomes are expected to contribute to 
academic knowledge in the field of AI-powered asset management, 
fostering further innovation. 

\vspace{0.3cm}
\subsubsection{Ethical considerations}
\vspace{0.2cm}
Ethically, the project will investigate issues related to data privacy, 
 transparency, and bias, which are critical in ensuring that 
automated systems operate fairly and without unintended consequences. 
These concerns are highlighted in the literature on AI ethics, which emphasizes 
the need for clear guidelines to mitigate risks associated with autonomous decision-making\citep{jobin2019global}.

\vspace{0.3cm}
\subsubsection{Sustainability, and social considerations}
\begin{figure}[htbp]
    \centering
    \includegraphics[width=0.48\linewidth]{targets.png}  % Replace with actual image file
    \caption{Sustainable Development Target 9.5 and 12.6}
    \label{fig:targets}  
\end{figure}

From a sustainability perspective, this research contributes to the United Nations Sustainable 
Development Goals (SDGs), specifically SDG 9, Industry, Innovation, and Infrastructure, and SDG 12, 
Responsible Consumption and Production, \citep{UN2030Agenda}. 
In relation to SDG 9, and more precisely 
target 9.5 as seen in Figure \ref{fig:targets}, the project seeks to enhance scientific 
research and upgrade the technological capabilities
 within industrial sectors. Similarly, under SDG 12 target 12.6 also shown in \ref{fig:targets}, 
 this project supports sustainable business practices by optimizing digital 
asset management. By enhancing asset categorization and retrieval, the system makes it easier 
for companies to track and store metrics. This dual focus ensures that the technological advancements 
proposed are not only efficient and innovative but also ethically sound and socially beneficial.

\vspace{0.3cm}
Further reflection will be revisited in Section 6.4.



\subsection{Goals}
% State the goal/goals of this degree project.
% The goal of this project is XXX. This has been divided into the following three sub-goals:
% 1. Subgoal \#1
% 2. Subgoal \#2
% 3. Subgoal \#3
% In addition to presenting the goal(s), you might also state what the deliverables and results of
% the project are.

% Note that in the literature study and even the alpha draft, 
% these are your expected goals, deliverables, and results – which may change over the course 
% of the project – hence you will revise this in the final report to describe what you actually 
% achieved, delivered, and produced as results.

The primary goal is evaluating the feasibility of a YOLO-powered DAM
system that has been fine-tuned using company-specific data, 
in comparison to the open-source solution ResourceSpace.
To achieve this, the project has been divided into the following three sub-goals:

\begin{enumerate} 
    
    \item \textbf{Dataset Development and Annotation:}
    Develop a robust methodology for collecting a domain-specific dataset that 
    accurately captures the visual and functional nuances of digital assets 
    in premium manufacturing. The annotation process will involve: 
    \begin{itemize} 
        \item Using bounding boxes to precisely delineate asset regions. 
        \item Assigning appropriate class labels using a standardized labeling schema 
        to ensure consistency and relevance to the manufacturing domain. 
    \end{itemize}
    
    This dataset will serve as the foundation for model fine-tuning.

    \item \textbf{Model Fine-Tuning and Optimization:}  
    Fine-tune a pre-trained YOLO model on the annotated dataset. 
    The objective is to enhance the model’s 
    accuracy in tagging, sorting, and categorizing.
    \begin{itemize}
        \item Adjusting hyperparameters and leveraging transfer learning techniques.
        \item Implementing regularization and validation strategies.
    \end{itemize}

    \item \textbf{Performance Benchmarking and Comparative Analysis:}  
    Benchmark the performance of the fine-tuned YOLO-based DAM system against a 
    conventional open-source DAM called ResourceSpace. Evaluation metrics will include:
    \begin{itemize}
        \item Asset categorization accuracy.
        \item Retrieval efficiency.
        \item Overall system usability.
    \end{itemize}
\end{enumerate}

A comparative analysis will be conducted to assess whether the customized 
system offers significant improvements over traditional solutions. 
Resulting in practical recommendations 
and guidelines for manufacturing SMEs 
considering the adoption of AI-powered DAM.


\subsection{Research Methodology}
This research employs a mixed-methods approach to address both the 
technical performance of the system and stakeholder perspectives. 
Mixed-methods research combines quantitative techniques 
(e.g., controlled experiments and statistical analyses) with 
qualitative techniques (e.g., semi-structured interviews and thematic analysis)
to provide a comprehensive evaluation of complex systems \citep{johnson2004mixed}.

Alternative methodologies—such as exclusively quantitative performance evaluations 
or purely qualitative case studies—were considered but ultimately rejected because 
they would not fully capture the multifaceted challenges of deploying an AI-powered 
system in a dynamic industrial environment.

\vspace{0.3cm}
\subsubsection{Design Science Approach}
\vspace{0.3cm}

% Introduce your choice of methodology/methodologies and method/methods – and the reason why
% you chose them. Contrast them with and explain why you did not choose other methodologies or
% methods. (The details of the actual methodology and method you have chosen will be given in
% Chapter 3. Note that in Chapter 3, the focus could be research strategies, data collection, data
% analysis, and quality assurance.)
% In this section you should present your philosophical assumption(s), research method(s), and
% research approach(es).

Grounded in a pragmatic philosophy that emphasizes practical impact and utility, 
this study adopts the design science research (DSR) paradigm. DSR is particularly 
well-suited for technology-driven projects because it promotes the iterative design, 
development, and rigorous evaluation of IT artifacts to solve 
real-world problems \citep{hevner2004design}. In this project, 
the YOLO-powered DAM system represents 
the artifact developed and refined through iteration.

\vspace{0.3cm}
\subsubsection{Quantitative and Qualitative Methods }
\vspace{0.3cm}
Controlled experiments will be conducted to measure key performance 
metrics—such as asset categorization accuracy, retrieval efficiency, 
and overall system usability. Statistical analysis w
ill be used to validate the improvements brought about by model fine-tuning, 
following best practices in empirical research \citep{creswell2014, yin2014case}.
Complementing this, qualitative methods will capture contextual insights and 
stakeholder perspectives. Semi-structured interviews and thematic analysis 
will be employed to understand user experiences and organizational challenges 
associated with implementing the DAM system. 
Moreover, to develop a standardized 
labeling schema for the dataset, a targeted collaboration with a designated 
expert from the company will be undertaken. This focused approach is 
preferred over a large-scale survey. Not all employees interact 
with digital assets and the expert can ensure domain-specific 
terminology is accurately captured and applied consistently during annotation.

\subsection{Delimitations}
% Describe the boundary/limits of your thesis project and what you are explicitly not going to do. This
% will help you bound your efforts – as you have clearly defined what is out of the scope of this
% thesis project. Explain the delimitations. These are all the things that could affect the study if they
% were examined and included in the degree project.
This thesis focuses exclusively on evaluating a YOLO-powered digital asset management 
system for premium manufacturing SMEs. The study is limited to a specific company’s 
environment and a predefined dataset.

The research investigates only the fine-tuning of an existing pre-trained YOLOv11 model. 
Training a model from scratch, which requires vast amounts of data and computational resources, 
is beyond the scope of this project.
Instead of conducting a large-scale survey, the study uses semi-structured interviews with key 
stakeholders—particularly a designated domain expert—to develop a standardized labeling schema. 

This focused approach is chosen because only a few employees directly manage digital assets.
The assessment will concentrate on technical performance indicators such as asset categorization accuracy, 
retrieval efficiency, and overall system usability. Broader issues such as integration with other enterprise systems 
and macroeconomic impacts are beyond the scope of this project.

\subsection{Structure of the thesis}
% Chapter 2 presents relevant background information about xxx. Chapter 3 presents the
% methodology and method used to solve the problem. …
% Exclude the first chapter , references, and appendix/appendices. 

This thesis is organized into the following main chapters, 
excluding the introductory chapter, references, and appendices; 
Chapter 2 provides the necessary background and reviews related work,
 establishing the context for DAM and identifying the key gaps this project addresses. 
 Chapter 3 outlines the methodology—including the design science approach, 
 mixed-methods strategy, data collection, experimental design, and evaluation 
 criteria—used to assess the system. 
 Chapter 4 details the implementation, covering system design, 
 model fine-tuning, dataset development, and the technical setup for testing. 
 Chapter 5 presents the results and analysis, discussing both quantitative 
 metrics and qualitative insights to evaluate whether the project’s goals have been met. 
 Finally, Chapter 6 summarizes the key findings, reflects on the limitations of the study, 
 and outlines potential directions for future work.




% --------------------------------------------------------------------------------------------------------------------------------------------
\newpage
% ===== Section 2: Background =====
\section{Background}
% This chapter provides basic background information about xxx. Additionally, this chapter describes
% xxx. The chapter also describes related work xxxx.
% What does a reader (another x student -- where x is your study line) need to know to understand
% your report?
% What have others already done? (This is the “related work”.) Explain what and how prior work /
% prior research will be applied on or used in the degree project /work (described in this thesis).
% Explain why and what is not used in the degree project and give valid reasons for rejecting the
% work/research.

% When you do your literature study, you should have a nearly complete Chapters 1, 2.

% You may also find it convenient to introduce the 
% future work section into your report early – so that you can 
% put things that you think about but decide not to do now into this section.

% Note that later you can move things between this future work section and what 
% you have done as you may change your mind about w
% hat to do now versus what to put off to future work.

\subsection{\mbox{Digital Asset Management}}
\cite{krogh2009} describes DAM as an essential framework for protecting, 
organizing, and prolonging the usability of digital files by emphasizing 
metadata, suitable file formats, and efficient workflows. As shown in Figure \ref{fig:4steg}, 
five interconnected stages—creation, management, distribution, archiving, and retrieval—collectively 
ensure that digital assets remain discoverable and relevant long after their initial production.

Although Krogh does not explicitly align his approach with the Resource-Based View (RBV), 
his emphasis on preserving assets as integral organizational resources parallels RBV’s 
tenet that competitive advantage relies on valuable, rare, inimitable, and non-substitutable 
(VRIN) capabilities \citep{barney1991}. By structuring DAM processes around rigorous 
metadata management, secure storage, and ongoing accessibility, organizations can treat 
their digital repositories as strategic assets, safeguarding long-term benefits that are 
difficult for competitors to replicate.

\begin{figure}[htbp]
    \centering
    \includegraphics[width=0.5\linewidth]{4steg.png}  % Replace with actual image file
    \caption{Illustrating the five main stages of DAM.}
    \label{fig:4steg}  
\end{figure}

\vspace{0.3cm}
\subsubsection{Technological Tools Demand Continuous Organizational Adaptation}
\vspace{0.3cm}

\cite{LOVE2019102930} identify a critical gap in the construction industry: knowing “why” 
to adopt digital technologies is relatively straightforward, but knowing “how” to translate 
technological potential into real value remains largely underexplored. Their case studies underscore 
the fact that digital transformation does not happen automatically; organizations must actively
 invest in processes such as benefits management and the development of a Business Dependency 
 Network (BDN) to realize tangible gains from their digital initiatives \citep{LOVE2019102930}.

 In a broader context, Hanelt et al. (2020) posit that digital transformation (DT) goes beyond 
any single disruptive episode; it is a continual, structural adjustment propelled by digital 
technologies. Their systematic review of 279 peer-reviewed articles frames DT across three 
dimensions—Contextual Conditions (e.g., technological advances, shifting consumer habits), 
Mechanisms (e.g., the innovative strategies organizations adopt), and Outcomes 
(e.g., changes to organizational structures and industry norms). By proposing 
a typology that spans technology impact, compartmentalized adaptation, systemic 
shift, and holistic co‐evolution, they challenge the idea of one-off change, 
advocating instead for an iterative, agile approach to transformation \citep{Haneltarticle}.

Taken together, these two perspectives highlight that while there is strong motivation to 
deploy new technologies (“why”), sustained, organization-wide benefits only materialize 
when there is a concerted effort to integrate, evaluate, and adapt these digital tools 
in an ongoing manner (“how”). Both studies imply that true success hinges on long-term 
structural and cultural shifts rather than static, one-off solutions.

\vspace{0.3cm}

that the promise of DAM
is not unlocked simply by adopting new technology but only when companies embrace two 
fundamental principles. First, that technology alone does not create value but must be accompanied by 
organizational process reengineering, and second, that the benefits of DAM 
are maximized only through continuous strategic governance to monitor and sustain its impact 


A missing perspective in

Nevertheless, some scholars argue that
resource possession alone does not guarantee successful 
digital transformation. 
\cite{Civelek2023} found no significant link be-
tween dynamic capabilities—a key aspect
of RBV that involves adapting, integrating,
and reconfiguring resources—and successful
digital transformation among Czech manu-
facturing SMEs. Their findings suggest that
merely possessing dynamic capabilities is
insufficient for digital transformation unless
supported by complementary factors such as
digital literacy and IT infrastructure matu-
rity.



% What about the benefits? A missing perspective in software engineering
% There are xxx characteristics that distinguish yyy from other information and communication
% technology (ICT) system, as shown in Figure 2-1. Table 2.1 summarizes these characteristics.




\ref{fig:lit_review} is an image 
\ref{tab:lit_review} is a table

\subsubsection{Major background area\#1\#1}
Recent studies have demonstrated the effectiveness of various AI techniques in image tagging. 
Zhang et al. (2019) showcased the application of convolutional neural networks (CNNs) for automatic 
image classification in DAM systems, achieving an accuracy of 92\% on a diverse dataset of digital assets

This work was further extended by Li 
and Chen (2020), who integrated attention mechanisms into CNNs, improving the model's ability to focus on 
salient features and increasing tagging accuracy to 95\%

The YOLO (You Only Look Once) algorithm has also been applied successfully in DAM contexts. 
Wang et al. (2021) demonstrated that YOLO-based models could perform real-time object detection and tagging 
in DAM systems, processing up to 30 images per second with an average precision of 88\%
This approach was particularly effective for identifying multiple objects within complex images, 
a common requirement in DAM applications.

Transformer-based models have recently gained traction in image tagging for DAM systems. A study by 
Rodriguez and Kim (2022) applied Vision Transformer (ViT) models to DAM image tagging, achieving 
state-of-the-art performance with an accuracy of 97\% on standard benchmarks
The authors noted that transformer models excelled in capturing long-range dependencies in images, 
leading to more nuanced and context-aware tagging.


While AI-powered image tagging offers significant benefits, it also presents several challenges. Data requirements 
pose a significant hurdle, as highlighted by Brown et al. (2020), who found that AI models required at 
least 10,000 labeled images per category for optimal performance in domain-specific DAM applications

Error rates and handling domain-specific content remain ongoing challenges. A comprehensive study by 
Thompson et al. (2021) analyzed error patterns in AI-powered image tagging across various industries, 
revealing that error rates increased significantly (up to 25\%) when dealing with highly specialized or technical imagery

To address this issue, Nguyen and Patel (2022) proposed a hybrid approach combining pre-trained models with 
domain-specific fine-tuning, reducing error rates by 40\% in niche industries such as medical imaging and aerospace engineerin

Despite these challenges, the benefits of AI-powered image tagging in DAM systems are substantial. A large-scale study by Garcia et al. (2023)
 across 500 organizations found that implementing AI-powered tagging led to a 60\% reduction in manual tagging time and 
 a 35\% improvement in asset discoverability


Entangled states are an important part of quantum cryptography, but also relevant in other
domains. This concept might be relevant for neutrinos, see for example [2].

\subsubsection{What is the YOLO model + how does it work from a higher level perspective}
\vspace{0.3cm}

Object detection algotithm. It locates object in an image
It is ine stage detection, it is fater than two stage. 
It is just a algoritm... so you have to do a lot around it?

\paragraph{Scheme}

% \begin{figure}[h]
%     \centering
%     \includegraphics[width=1.2\linewidth]{YOLOSTEP.png}  % Replace with actual image file
%     \caption{YOLOv11 performance comparison (Ultralytics Inc., 2025).}
%     \label{fig:Scheme}  
% \end{figure}

\begin{figure}[h]
    \centering
    \includegraphics[width=1\linewidth]{YOLOARC.png}  % Replace with actual image file
    \caption{The architecture of YOLOv11, illustrating its three main components: Backbone, Neck, and Head.}
    \label{fig:Arc}  
\end{figure}

The YOLOv11 model follows the standard three-part structure of the YOLO family: 
Backbone, Neck, and Head, as shown in Figure \ref{fig:Arc}.
According to Hidayatullah et al. (2025), the Backbone extracts features using 
convolutional layers and downsampling, generating hierarchical feature maps.
The Neck refines these features through the SPPF block for multi-scale detection 
and the C2PSA module to enhance the recognition of small and occluded objects. 
Upsampling and feature concatenation further improve resolution and information 
retention. Finally, the Head produces the model’s output, predicting class probabilities 
and bounding boxes across three detection layers (small, medium and large), 
each specialized for different object sizes \citep{hidayatullah2025yolov8yolo11comprehensivearchitecture}.
\vspace{0.3cm}

and the Head 
Create bounding boxes and pair it to a class.

step 1. Overlay image with a grid in size sxs 
Each grid cell produces 2 things:
- 1. a set of bounding boxes centered on a point inside the grid
 with conficence scores that an object exists inside each bounding box 

 - 2 a class probability map for each cell: which tells what object class is most 
 likely to be in that cell given that an object exists within the cell.

It combines this info to yeild the object detections 

\paragraph{Architecture}


\paragraph{Loss Function}


\subsubsection{The YOLO model}
\vspace{0.3cm}
As demonstrated in table \ref{tab:yolo_versions} the YOLO series has
 evolved significantly since its inception, introducing progressive improvements
  in object detection, computational efficiency, and feature extraction. 
YOLOv11 is the best choice for the project due to its superior accuracy, 
efficiency, and versatility. As Khanam and Hussain (2024) highlight, 
its architectural upgrades enhance feature extraction while minimizing 
computational costs, making it ideal for real-time applications requiring 
both speed and precision \citep{khanam2024yolov11overviewkeyarchitectural}.

\vspace{0.3cm}
Beyond object detection, YOLOv11 supports instance segmentation, 
pose estimation, and oriented object detection, offering greater adaptability 
to the project’s needs. Its optimized balance of accuracy and processing speed 
ensures strong performance across different computing environments, from edge 
devices to high-performance systems, making it the most effective solution

\begin{table}[t]
    \centering
    {\small
    \begin{tabularx}{\linewidth}{|c|X|}
        \hline
        \textbf{Release} & \textbf{Capabilities} \\
        \hline
        \begin{tabular}[t]{@{}c@{}}
            \textbf{V1}\\
            {\scriptsize Darknet}\\
            {\tiny JUN 2015}
        \end{tabular}
        &
        A single-stage object detector with basic classification
        \citep{redmon2016you}.
        \\
        \hline

        \begin{tabular}[t]{@{}c@{}}
            \textbf{V2}\\
            {\scriptsize Darknet}\\
            {\tiny DEC 2016}
        \end{tabular}
        &
        Object detection. Darknet-19 architecture, anchor boxes, and higher resolution inputs
        \citep{redmon2016yolo9000betterfasterstronger}.
        \\
        \hline

        \begin{tabular}[t]{@{}c@{}}
            \textbf{V3}\\
            {\scriptsize Darknet}\\
            {\tiny MAR 2018}
        \end{tabular}
        &
        Object detection. Darknet-53 network \& multi-scale predictions for varying object sizes.
        \citep{redmon2018yolov3}.
        \\
        \hline

        \begin{tabular}[t]{@{}c@{}}
            \textbf{V4}\\
            {\scriptsize Darknet}\\
            {\tiny APR 2020}
        \end{tabular}
        &
        Object detection. Basic object tracking with BCSPDarknet53 and SPP.
        \citep{bochkovskiy2020yolov4}.
        \\
        \hline

        \begin{tabular}[t]{@{}c@{}}
            \textbf{V5}\\
            {\scriptsize PyTorch}\\
            {\tiny JUN 2020}
        \end{tabular}
        &
        Object detection. Basic instance segmentation. PyTorch framework,
        multi-GPU support, and exports \citep{ultralytics2024yolov5}.
        \\
        \hline

        \begin{tabular}[t]{@{}c@{}}
            \textbf{V6}\\
            {\scriptsize PyTorch}\\
            {\tiny SEP 2022}
        \end{tabular}
        &
        Object detection \& instance segmentation. Reparameterizable backbone for model scaling.
        \citep{li2022yolov6}.
        \\
        \hline

        \begin{tabular}[t]{@{}c@{}}
            \textbf{V7}\\
            {\scriptsize PyTorch}\\
            {\tiny JUL 2022}
        \end{tabular}
        &
        Object detection, tracking \& instance segmentation.
        \citep{wang2022yolov7}.
        \\
        \hline

        \begin{tabular}[t]{@{}c@{}}
            \textbf{V8}\\
            {\scriptsize PyTorch}\\
            {\tiny JAN 2023}
        \end{tabular}
        &
        Object detection, tracking, instance segmentation, panoptic segmentation, keypoint estimation.
        NVIDIA GPUs, Jetson systems and macOS.
        \citep{ultralytics2025yolov8}
        \\
        \hline

        \begin{tabular}[t]{@{}c@{}}
            \textbf{V9}\\
            {\scriptsize PyTorch}\\
            {\tiny FEB 2024}
        \end{tabular}
        &
        Object detection \& instance segmentation.
        PGI for better gradient reliability.
        GELAN network \citep{wang2024yolov9}.
        \\
        \hline

        \begin{tabular}[t]{@{}c@{}}
            \textbf{V10}\\
            {\scriptsize PyTorch}\\
            {\tiny MAY 2024}
        \end{tabular}
        &
        Object detection \& NMS-free training
        \citep{wang2024yolov10}
        \\
        \hline

        \begin{tabular}[t]{@{}c@{}}
            \textbf{V11}\\
            {\scriptsize PyTorch}\\
            {\tiny SEP 2024}
        \end{tabular}
        &
        Object detection, instance segmentation, pose estimation \& oriented object detection (OBB).
        \citep{UltralyticsYOLO11}.
        \\
        \hline

    \end{tabularx}
    }
    \caption{Summary of YOLO Model Evolution}
    \label{tab:yolo_versions}
\end{table}


\begin{figure}[h]
    \centering
    \includegraphics[width=1\linewidth]{YOLOV11.png}  % Replace with actual image file
    \caption{YOLOv11 performance comparison (Ultralytics Inc., 2025).}
    \label{fig:bench}  
\end{figure}

The selection of YOLOv11 for the project is driven by its superior architectural 
enhancements, versatile task support, and optimized balance between accuracy and 
efficiency. 
  Each version has incorporated refinements aimed at enhancing real-time performance, 
  with YOLOv11 representing the most advanced iteration to date \citep{khanam2024yolov11overviewkeyarchitectural}.
\vspace{0.3cm}

Computational methods are increasingly used as a third method of carrying out scientific
investigations. For example, computational experiments were used to find the amount of wear in a
polyethylene liner of a hip prosthesis in [3].

\subsection{Major background area\#2}
The application of AI-powered image tagging in DAM systems extends beyond large corporations 
to small and medium-sized enterprises (SMEs), particularly in premium manufacturing sectors. A case study 
by Hoffmann and Schulz (2022) examined the implementation of AI-powered DAM in a high-end carpentry 
company similar to Veermakers
The study found that AI-assisted tagging improved product catalog management 
efficiency by 45\% and reduced time-to-market for new designs by 30\%.

However, Chen et al. (2023) noted that SMEs in specialized manufacturing often 
face unique challenges in adopting AI-powered DAM systems, including limited datasets 
and highly specific visual content. 
To address these issues, the authors proposed a transfer learning approach, adapting pre-trained 
models to domain-specific tasks with minimal additional data, achieving a 75\% reduction in required 
training data while maintaining 90\% of the original accuracy.

While academic research has made significant strides in advancing AI-powered image tagging techniques, 
commercial implementations often lag behind in adopting cutting-edge methods. A comprehensive survey by Martinez 
and Lee (2022) of 50 leading DAM vendors revealed that only 30\% had implemented transformer-based models, despite 
their superior performance in academic studies
The authors attributed this gap to factors such as implementation complexity, computational requirements, and the need 
for backward compatibility with existing systems.




\subsubsection{Major background area\#2\#1}
The integration of AI-powered image tagging in DAM systems raises important ethical, societal, and legal considerations.
 Privacy concerns are paramount, as highlighted by a study by Johnson and Smith (2022), which found that 35\% of 
 automatically generated tags in a sample of 10,000 images contained potentially sensitive information22. The authors 
 emphasized the need for robust privacy-preserving techniques in AI-powered DAM systems.
 Algorithmic bias presents another significant challenge. Research by Park et al. (2023) revealed systematic biases 
 in AI-generated tags across gender, ethnicity, and age dimensions, with error rates up to 20\% higher for underrepresented groups
 This study underscores the importance of diverse and representative training data in mitigating bias in AI-powered DAM systems.
\subsubsection{Major background area\#2\#2}
The potential impact on employment is also a concern. While Garcia et al. (2023) found that AI-powered tagging led to significant 
efficiency gains, they also noted a 15\% reduction in human tagging roles across surveyed organizations
However, the same study observed a 10\% increase in higher-skilled positions related to AI model management and quality 
assurance, suggesting a shift rather than a net loss in employment.


\subsection{Related work}
\subsubsection{Major related work}
Do not use the title of the paper/book/… as the title of the section. Instead summarize what the contribution of this work is in your own words.

Geo-distributed data centers are increasingly used to provide increased availability and reduce
latency; however, the physically nearest data center may not be the best choice as shown by Kirill
Bogdanov, et al. in their paper “The Nearest Replica Can Be Farther Than You Think” [4].
Exploring decentralized approaches to AI model training, allowing organizations to collaborate on improving tagging accuracy while preserving data privacy.

\subsubsection{Major related work}
Carrier clouds have been suggested as a way to reduce the delay between the users and the cloud
server that is providing them with content. However, there is a question of how to find the available
resources in such a carrier cloud. One approach has been to disseminate resource information using
an extension to OSPF-TE, see Roozbeh, Sefidcon, and Maguire [5].

\subsubsection{Minor related work}
Do not use the title of the paper/book/… as the title of the section. Instead summarize what the contribution of this work is in your own words.

\subsection{Summary}
It is nice to bring this chapter to a close with a summary. For example, you might include a table that summarizes the ideas of others and the advantages and disadvantages of each – so that later you can compare your solution to each of these. This will also help guide you in defining the metrics that you will use for your evaluation.


% ===== Section 3: Methodology =====
\section{<Engineering-related content, Methodologies and Methods>
Use a self-explaining title}

The contents and structure of this chapter will change with your choice of methodology and methods.
For example, if you have implemented an artifact, what did you do and why? How will your evaluate it.


Describe the engineering-related contents (preferably with models) and the research methodology
and methods that are used in the degree project.
Give a theoretical description of the scientific or engineering methodology are you going to use
and why have you chosen this method. What other methods did you consider and why did you reject
them.
In this chapter, you describe what engineering-related and scientific skills you are going to
apply, such as modeling, analyzing, developing, and evaluating engineering-related and scientific
content. The choice of these methods should be appropriate for the problem. Additionally, you
should be consciousness of aspects relating to society and ethics (if applicable). The choices should
also reflect your goals and what you (or someone else) should be able to do as a result of your
solution - which could not be done well before you started.
The purpose of this chapter is to provide an overview of the research method used in this thesis.
Section 3.1 describes the research process. Section 3.2 details the research paradigm. Section 3.3
focuses on the data collection techniques used for this research. Section 3.4 describes the
experimental design. Section 3.5 explains the techniques used to evaluate the reliability and validity
of the data collected. Section 3.6 describes the method used for the data analysis. Finally, Section 3.7
describes the framework selected to evaluate xxx.

\subsection{Research Process}
Image of: steps conducted to do the research 
Fig: research processes


\subsection{Research Paradigm}

\subsection{Data Collection}
(This should also show that you are aware of the social and ethical concerns that might be relevant
to your data collection method.)

\subsubsection{Sampling}
1. Aa
2. Bb
3. Cc

\subsubsection{Sample Size}

\subsubsection{Target Population}


\subsection{Experimental design/Planned Measurements}


\subsubsection{Test environment/test bed/model}
Describe everything that someone else would need to reproduce your test environment/test
bed/model/…

\subsubsection{Hardware/Software to be used}


\subsection{Assessing reliability and validity of the data collected}

\subsubsection{Reliability}
How will you know if your results are reliable?

\subsection{Validity}
How will you know if your results are valid?



\subsection{Planned Data Analysis}

\subsubsection{Data Analysis Technique}
\subsubsection{Software Tools}


\subsection{Evaluation framework}




% ===== Section 4: WHAT has been Done =====
\section{[What you did – Choose your own chapter title to describe this]}
What have you done? How did you do it? What design decisions did you make? How did what you
did help you to meet your goals?

\subsection{Hardware/Software design …/ModelSimulation model parameters/…}
Figure 4-1 shows a simple icon for a home page. The time to access this page when served will be
quantified in a series of experiments. The configurations that have been tested in the test bed are
listed in Table 4-1.
\begin{figure}[htbp]
    \centering
    \includegraphics[width=0.4\linewidth]{kthLogga.png}  % Replace with actual image file
    \caption{An example figure in Section.}
    \label{fig:ldone}  
\end{figure}

\begin{table}[htbp]
    \centering
    \begin{tabular}{|c|c|}
        \hline
        Column 1 & Column 2 \\
        \hline
        Data 1 & Data 2 \\
        Data 3 & Data 4 \\
        \hline
    \end{tabular}
    \caption{An example table in Section.}
    \label{tab:done}  
\end{table}



\ref{fig:ldone} is an image 
\ref{tab:done} is a table


\subsection{Implementation …/Modeling/Simulation/…}






% ===== Section 5: Results and analysis  =====
\section{Results and Analysis}
In this chapter, we present the results and discuss them.

Keep in mind: How you are going to evaluate what you have done? What are your metrics?
Analysis of your data and proposed solution
Does this meet the goals which you had when you started?

\subsection{Major results}
Some statistics of the delay measurements are shown in Table 5-1.
The delay has been computed from the time the GET request is received until the response is
sent.

\begin{table}[htbp]
    \centering
    \begin{tabular}{|c|c|}
        \hline
        Column 1 & Column 2 \\
        \hline
        Data 1 & Data 2 \\
        Data 3 & Data 4 \\
        \hline
    \end{tabular}
    \caption{An example table in Section}
    \label{tab:res}  
\end{table}

\ref{tab:res} is a table

\subsection{Reliability Analysis}
LALALA

\subsection{Validity Analysis}
LALALA


\subsection{Discussion}





% ===== Section 6: CONCLUSION  =====
\section{Conclusions and Future work}
<<Add text to introduce the subsections of this chapter.>>

\subsection{Conclusions}
Describe the conclusions (reflect on the whole introduction given in Chapter 1).
Discuss the positive effects and the drawbacks.
Describe the evaluation of the results of the degree project.
Did you meet your goals?
What insights have you gained?
What suggestions can you give to others working in this area?
If you had it to do again, what would you have done differently?

\subsection{Limitations}
What did you find that limited your efforts? What are the limitations of your results?

\subsection{Future work}
Describe valid future work that you or someone else could or should do.
Consider: What you have left undone? What are the next obvious things to be done? What hints
can you give to the next person who is going to follow up on your work?

\subsection{Reflections}
What are the relevant economic, social, environmental, and ethical aspects of your work?





% ===== Section 7: References =====

\addcontentsline{toc}{section}{References}  % Lägg till i innehållsförteckningen
\bibliographystyle{apalike}
\nocite{*}
\bibliography{references} % Hänvisar till din references.bib-fil


% \bibliographystyle{apalike}
% \nocite{*}
% \bibliography{references} % Ensure your references.bib file includes all cited sources




% ===== Appendices =====
\newpage
\onecolumn
\appendix  % This marks the start of the appendices
\phantomsection
\addcontentsline{toc}{section}{Appendices} % Adds "Appendices" to Table of Contents
\section*{Appendices} % Title for Appendices
\renewcommand{\thesubsection}{\Alph{subsection}} % Number subsections as A, B, C

\section{Appendix A: Example Appendix Title}
\label{appendix:A}
This is an example appendix entry. You can include figures, tables, or additional details relevant to your research.

\begin{figure}[htbp]
    \centering
    \includegraphics[width=0.4\linewidth]{kthLogga.png}  % Replace with actual image file
    \caption{An example figure in Appendix A.}
    \label{fig:appendixA}  
\end{figure}

\begin{table}[htbp]
    \centering
    \begin{tabular}{|c|c|}
        \hline
        Column 1 & Column 2 \\
        \hline
        Data 1 & Data 2 \\
        Data 3 & Data 4 \\
        \hline
    \end{tabular}
    \caption{An example table in Appendix A.}
    \label{tab:appendixA}  
\end{table}

\newpage
\section{Appendix B: Another Appendix Example}
\label{appendix:B}
You can continue adding appendices in a similar manner.

IEEE Editorial Style Manual: 
	

\end{document}


