\documentclass[a4paper,12pt,twocolumn]{article}


\usepackage[utf8]{inputenc} % Support for UTF-8 encoding
\usepackage[T1]{fontenc}    % Better font rendering
\usepackage{graphicx}       % For including images
\usepackage{amsmath}        % For mathematical formulas
\usepackage{hyperref}       % For clickable links in the PDF
\usepackage{geometry}       % To adjust margins
\geometry{margin=2.5cm}
\usepackage{titlesec}       % For custom section headings
\usepackage{fancyhdr}       % For header and footer customization
\usepackage{natbib}         % For in-text citations (Harvard style)
\usepackage{lmodern}        % Improved font quality
\usepackage{setspace}       % For adjusting line spacing
\usepackage{ragged2e}       % Fix justification issues
\usepackage[protrusion=true,expansion=true]{microtype} % Better text spacing


% ---- Fix fancyhdr warning: set proper headheight ----
\setlength{\headheight}{14.5pt}

% ---- Header settings ----
\pagestyle{fancy}
\fancyhf{}
\lhead{KEX Project}
\rhead{KTH}
\cfoot{\thepage}

\begin{document}

% ---- Suppress page numbering on title page to avoid duplicate "page.1" link ----
\clearpage
\thispagestyle{empty}  % Removes the page number but avoids identifier conflicts

\setlength{\headheight}{14.5pt}
% ---- Title Page ----
\begin{titlepage}
    \centering
    \includegraphics[width=0.2\textwidth]{kthLogga.png}\\[1cm]
    {\large BACHELOR'S THESIS IN COMPUTER SCIENCE AND INDUSTRIAL ECONOMICS}\\[0.5cm]
    {\large UNDERGRADUATE LEVEL 15 CREDITS}\\[3cm]
    {\Huge \textbf{A Comparative Evaluation of Open-Source Digital Asset Management Systems}}\\[0.5cm]
    {\Large Exploring Organizational and Marketing Criteria for Process and Marketing Innovation in SMEs}\\[1cm]
    \vfill
    {\Large \textbf{ELLA KARLSSON}}\\[1cm]
    \vfill
    {\large School of Industrial Engineering and Management}\\
    {\large Royal Institute of Technology (KTH)}\\
\end{titlepage}

% ---- Ensure Table of Contents is in Single Column ----
\newpage
\onecolumn
\tableofcontents
\newpage

% ---- Switch Back to Two-Column Layout for Main Content ----
\twocolumn
\pagenumbering{arabic}

% ===== Apply Spacing Fixes =====
\RaggedRight   % Disable full justification to avoid stretched text
\sloppy        % Allow better text breaking to prevent large spaces
\hyphenpenalty=500
\tolerance=1000

% ===== Section 1: Introduction =====
\section{Introduction}
\subsection{Research Question}
The research question investigated in this pre-study is:

\begin{quote}
    \textit{"To what extent does DAM adoption contribute to improved operational efficiency and strategic positioning in a premium manufacturing company?"}
\end{quote}

\subsection{Connection to the Technical Project}
The technical aspect of this study examines the adoption of an open-source DAM system versus 
the development of a tailored solution, with a focus on role-based access control, security, 
logging, and usability. 

\vspace{0.3cm} 
However, technological advancements alone do not guarantee successful integration. To complement this,
the business perspective assesses the organizational and strategic impact after selecting the preferred DAM system. Specifically:
\begin{itemize}
    \item \textbf{Process Impact:} Assessing whether the DAM system has enhanced workflow efficiency, minimized errors, and reduced redundant tasks.
    \item \textbf{Organizational Adoption:} Evaluating the ease of employee adaptation, the necessity of training, and any role adjustments required for successful implementation.
    \item \textbf{Strategic Impact:} Determining whether the DAM system strengthens brand consistency, improves customer engagement, and supports scalability as the company expands.
\end{itemize}

\vspace{0.3cm} 
\subsection{Societal Impact}
Digital transformation has a significant impact on small and medium-sized enterprises (SMEs). SMEs account for approximately 
60\% of total turnover and value-added contributions in Sweden’s private sector, employing around 65\% of the workforce \citep{tillvaxtverket2021}.

\vspace{0.3cm}
The adoption of DAM systems is an integral part of this transformation, 
improving operational efficiency and reducing manual work,
which contributes to broader economic growth. A cost-benefit analysis of 319 SMEs 
found that digital transformation enhances organizational resilience, reduces 
operational costs, and improves long-term scalability \citep{teng2022}.

% ===== Reset Formatting for Other Sections =====
\justifying
\fussy

% ===== Section 2: Theoretical Framework and Previous Studies =====
\section{Theoretical Framework and Previous Studies}
This section builds upon existing literature and theoretical frameworks to analyze how DAM adoption influences 
strategic decision-making, cost structures, and competitive positioning in SMEs. A particular emphasis is placed 
on the interplay between technological capabilities, organizational governance, and leadership dynamics,
as these factors shape the effectiveness of DAM systems in practice.


\subsection{The Resource-Based View}
The Resource-Based View (RBV) suggests that organizations derive competitive advantage by leveraging valuable, rare, inimitable, 
and non-substitutable (VRIN) resources \cite{barney1991}. DAM systems fulfill these criteria by providing centralized governance,
metadata standardization, and automation, transforming digital assets into strategic resources \cite{Chumphong2020}.

\vspace{0.3cm}
Empirical findings support this perspective. \cite{Chumphong2020} report that SMEs achieving full DAM integration 
demonstrate 23\% higher operational resilience during market disruptions compared to those with only partial implementation. 
This aligns with RBV’s emphasis on resource orchestration and its role in maintaining competitiveness.

\vspace{0.3cm}
Nevertheless, some scholars argue that resource possession alone does not guarantee successful digital transformation. 
\cite{Civelek2023} found no significant link be-
tween dynamic capabilities—a key aspect
of RBV that involves adapting, integrating,
and reconfiguring resources—and successful
digital transformation among Czech manu-
facturing SMEs. Their findings suggest that
merely possessing dynamic capabilities is
insufficient for digital transformation unless
supported by complementary factors such as
digital literacy and IT infrastructure matu-
rity.

\subsection{Simons’ Levers of Control}
\cite{simons1995} Levers of Control (LOC) framework is a central model for understanding how organizations 
balance control and innovation through management control systems (MCS). 
There are diagnostic control systems, which are used to monitor and ensure that operations 
align with set objectives, and interactive control systems, which facilitate strategic discussions and adaptation to 
changing market conditions. In a digital context, these control mechanisms become crucial for how organizations effectively 
implement systems.

\subsubsection{Diagnostic Control Systems}
By implementing DAM, organizations can systematically monitor resource usage and efficiency, reducing redundancies 
and improving workflows. Previous research indicates that such systems can reduce unnecessary tasks by up to 34\% through automated version management 
and asset tracking \cite{Mladenova2024}. 

\vspace{0.3cm}
\cite{wernerfelt1984} lays the foundation for resource-based theory, suggesting that firms should focus on acquiring and leveraging valuable resources 
rather than just competing on products or market positioning. The idea that DAM enhance governance and long-
term decision-making can be linked to his argument that a firm’s internal resources
dictate its competitive position. \cite{barney1991} builds upon this idea by highlighting the importance of VRIN resources that are difficult to imitate or substitute. 
DAM can play a key role in creating such competitive advantages by protecting and optimizing an organization’s digital content management. 
By gathering data on usage patterns and performance, DAM can also contribute to a more data-driven governance approach, strengthening long-term strategic decision-making \cite{wernerfelt1984}.

\subsubsection{Interactive Control Systems}
Technological advancements necessitate a continuous reassessment of strategies and work processes. DAM can serve as an interactive control 
tool by integrating digital assets into strategic processes and facilitating collaboration across organizational boundaries.
\cite{teece1997} argue that companies require dynamic capabilities 
to adapt to rapid environmental changes, and DAM systems can serve as such a capability by facilitating knowledge sharing and cross-functional collaboration.
\cite{Mladenova2024} found that companies implementing DAM experienced a 19\% faster time-to-market for new products, 
suggesting that it contributes to increased organizational flexibility and innovation capabilities.

\vspace{0.3cm}
However, for DAM to function as interactive control, management must actively engage in its use and foster a culture where digital 
tools are seen as integral to the company’s strategic development. 
\cite{simons1995} emphasizes that control systems must evolve alongside technological adoption and that management plays a central 
role in ensuring that new technology is integrated in a way that supports both control and innovation. 
 \cite{eisenhardt2000} furhter stress that technological resources alone do not provide a competitive advantage
 unless combined with organizational capabilities that enable adaptation and change

\subsection{Leadership dynamics}
This review extends prior research by addressing the underexplored role of leadership styles in mediating DAM success. 
\cite{Civelek2023} emphasizes the importance of joint ventures with IT firms in SME digital transformation, but does not explicitly examine how leadership styles influence 
DAM adoption. This gap suggests that existing frameworks may not fully account for the managerial processes that enable or hinder DAM integration within SMEs.

\subsection{Innovation Theory}
This study extends the discussion on digital transformation in SMEs by integrating the innovation classification outlined in the Oslo Manual.
Process innovation refers to the introduction of significantly improved production or delivery methods, including advancements in techniques, software, and organizational workflows \cite{oecd2018oslo}. 

% ===== Section 3: Research Methodology =====
\section{Research Methodology}
Undersöka om förändringarna är ”nyttiga” och ”hållbara” ur flera perspektiv:
1. Intern process (är arbetssättet väsentligt förbättrat och mer effektivt?)
2. Extern marknad (leder det till nya eller förbättrade sätt att nå och engagera kunder?)
3. Hållbarhet och tillväxt (kan lösningen skalas upp till nya marknader eller segment?)

\subsection{Literature Review}
A systematic literature review will be conducted to identify key factors influencing DAM adoption in SMEs. Topics include:
\begin{itemize}
    \item Digital transformation in SMEs
    \item Organizational change management
    \item Process and marketing innovation strategies
\end{itemize}

\subsection{Workshops and Interviews}
Workshops will be held with industry stakeholders, including designers, project managers, and business executives, to assess their needs and expectations. Interviews will be conducted to explore:
\begin{itemize}
    \item Existing workflow challenges
    \item Perceived value of DAM systems
    \item Business considerations influencing adoption
\end{itemize}

\subsection{Benchmarking}
A comparative analysis of DAM adoption in similar industries will be conducted, identifying best practices and potential pitfalls.

\subsection{Economic and Organizational Impact Analysis}
The study will evaluate:
\begin{itemize}
    \item \textbf{Quantitative factors:} Cost savings, efficiency improvements, and return on investment (ROI).
    \item \textbf{Qualitative factors:} Changes in collaboration dynamics, decision-making processes, and marketing strategies.
\end{itemize}

% ===== Section 4: Discussion =====
\section{Discussion}
\subsection{Suitability of the Theoretical Framework}
The Oslo Manual framework provides a structured way to classify and analyze innovation. However, its broad definitions may need refinement when applied to specific SME contexts.

\subsection{Expected Findings}
The study anticipates that the primary challenges in DAM adoption will be:
\begin{itemize}
    \item Resistance to change within SMEs.
    \item Need for a clear return on investment to justify adoption.
    \item Importance of a user-friendly design to ensure full adoption and utilization.
\end{itemize}

These findings will be validated through empirical data collection.

% ===== Section 5: Conclusion =====
\section{Conclusion}
The **industrial economics perspective** enriches the **technical development** of DAM systems by identifying **critical business considerations**.  
This pre-study establishes a foundation for further research into the **organizational and strategic factors** necessary for successful DAM adoption in **SMEs operating within premium manufacturing**.

% ===== Section 6: References =====
\section{References}
\bibliographystyle{apalike}
\bibliography{references} % Ensure your references.bib file includes all cited sources

\end{document}