\documentclass[a4paper,12pt,twocolumn]{article}

\usepackage[utf8]{inputenc} % Stöd för svenska tecken
\usepackage[T1]{fontenc}    % Bättre font-rendering
\usepackage{graphicx}       % För bilder
\usepackage{amsmath}        % För matematiska formler
\usepackage{hyperref}       % Klickbara länkar i PDF
\usepackage{geometry}       % Justering av marginaler
\geometry{margin=2.5cm}
\usepackage{titlesec}       % Anpassning av sektioner
\usepackage{fancyhdr}       % Sidhuvud och sidfot
\usepackage{natbib}         % Löpande referenser (Harvard-stil)
\usepackage{lmodern}        % Bättre typsnitt
\usepackage{setspace}       % För att justera radavstånd

% ---- Fix fancyhdr warning: set proper headheight ----
\setlength{\headheight}{12.5pt}

% ---- Sidhuvud ----
\pagestyle{fancy}
\fancyhf{}
\lhead{KEX-arbete}
\rhead{KTH}
\cfoot{\thepage}

\begin{document}

% ---- Suppress page numbering on title page to avoid duplicate "page.1" link ----
\pagenumbering{gobble}
\begin{titlepage}
    \centering
    \includegraphics[width=0.2\textwidth]{kthLogga.png}\\[1cm]
    {\large KANDIDATEXAMENSARBETE INOM DATATEKNIK OCH INDUSTRIELL EKONOMI}\\[0.5cm]
    {\large GRUNDNIVÅ 15 HP}\\[3cm]
    {\Huge \textbf{Demokratisering av hållbarhetsdata:}}\\[0.5cm]
    {\Large Designöverväganden för Business Intelligence-verktyg i Greentech-företag}\\[1cm]
    \vfill
    {\Large \textbf{ELLA KARLSSON}}\\[1cm]
    \vfill
    {\large Skolan för Elektroteknik och Datavetenskap}\\
    {\large Kungliga Tekniska Högskolan (KTH)}\\
    %{\large \today}\\
\end{titlepage}
\clearpage

% ---- Resume normal page numbering ----
\pagenumbering{arabic}

% Abstract
\begin{abstract}
Två trender som blivit alltmer påtagliga de senaste åren är ökade datamängder och ökat fokus på klimatförändringar. Denna rapport undersöker hur hållbarhetsdata kan integreras i Business Intelligence (BI) för greentech-företag.
\end{abstract}

\tableofcontents
\newpage

\section{Introduction}
Här beskriver du bakgrunden till ditt arbete och dess syfte, med referenser enligt \citet{author2025} eller \citep{author2025}.

\section{Method}
Beskriv hur du genomför din undersökning, vilka metoder du använder och varför. Använd \citet{example2025} för att nämna författaren i texten eller \citep{example2025} för att ha referensen inom parentes.

\section{Resultat}
Presentera de resultat du har fått fram.

\section{Diskussion}
Analysera och diskutera resultaten i relation till tidigare forskning och teorier.

\section{Slutsatser}
Sammanfatta de viktigaste insikterna och eventuella framtida forskningsmöjligheter.

\section{Referenser}
\bibliographystyle{apalike}
\bibliography{referenser} % referenser.bib should contain author2025 and example2025

\end{document}
